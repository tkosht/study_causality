\usepackage[japanese]{babel}

\usepackage{fontspec}

%欧文ローマン体の設定
\setmainfont[Scale=MatchUppercase]{TeX Gyre Termes} %いわゆるTimes New Roman

%欧文サンセリフ体の設定
\setsansfont[Scale=1]{TeX Gyre Heros} %いわゆるHelveticaもしくはArial

%欧文タイプライタ体の設定
\setmonofont[Scale=MatchLowercase]{zcoN} %Courierの横幅がすっきりした書体

%数学モードで使う書体の設定
%\usepackage{mathtools}
%\usepackage{amsmath}
%\usepackage{mathspec}
%\usepackage{unicode-math} %勝手に読み込まれるmathspecパッケージと相性が悪いのでロードしない
%\setmathfont[Scale=MatchUppercase]{Garamond-math.otf}  %

%\setmathfont(Digits,Latin,Greek)[Scale=MatchUppercase]{TeX Gyre Termes}
%\setmathfont[Scale=MatchUppercase]{TeX Gyre Termes}
\setmathfont[Scale=MatchUppercase]{TeX Gyre Termes Math} %Beamerでは適用できるが、article系統では適用できない?

\setmathrm{TeX Gyre Termes}
%https://tex.stackexchange.com/questions/11058/how-do-i-change-the-math-italic-font-in-xetex-fontspec

%日本語フォント設定のために読み込む
\usepackage{zxjatype}
%\usepackage[deluxe]{otf} %platexが必要なのでRmarkdown単体では使えない

\setjafontscale{1} %日本語フォントの大きさを、欧文フォントと釣り合うようにする%

%日本語明朝体の設定(太字も設定)
\setjamainfont[BoldFont=SourceHanSerifJP-Bold.otf]{SourceHanSerifJP-Light.otf}

%日本語ゴシック体の設定(太字も設定)
\setjasansfont[Scale=1,BoldFont=SourceHanSansJP-Medium.otf]{SourceHanSansJP-Light.otf}

%日本語タイプライタ体の設定
\setjamonofont{SourceHanSansJP-Normal.otf}

